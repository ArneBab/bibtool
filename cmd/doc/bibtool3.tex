%%*** bibtool3.tex ************************************************************
%%
%% This file is part of BibTool.
%% It is distributed under the Creative Commons Attribution-Share
%% Alike 3.0 License.
%%
%% (c) 1995-2016 Gerd Neugebauer
%%
%% Net: gene@gerd-neugebauer.de
%%
%%-----------------------------------------------------------------------------
%% Usage:  latex     bibtool
%%         bibtex    bibtool
%%         latex     bibtool
%%         makeindex -s bibtool.ist bibtool
%%         latex     bibtool
%%*****************************************************************************

\documentclass[11pt,a4paper]{scrbook}

\usepackage{bibtool-doc}

%%%*****************************************************************************
%%
%% This file is part of BibTool.
%% It is distributed under the GNU General Public License.
%% See the file COPYING for details.
%%
%% (c) 1995-2004 Gerd Neugebauer
%%
%% Net: gene@gerd-neugebauer@sdm
%%
%%*****************************************************************************
\newcommand\LIBDIR{/usr/local/lib/BibTool}
\newcommand\Year{2003}
\newcommand\Version{2.47}

\newcommand\Version{3}

\hypersetup{pdftitle={BibTool Manual}}
\hypersetup{pdfauthor={Gerd Neugebauer}}
\hypersetup{pdfsubject={Version \Version}}

\makeindex

\DeclareFontShape{OT1}{cmss}{m}{it}{<-> ssub * cmss/m/sl}{}

\definecolor{keyword}{rgb}{0,.5,0}
\definecolor{bibtex-bg}{rgb}{0.9607843137,0.9019607843,0.8235294118}
\usepackage{listings}
\lstdefinelanguage{BibTeX}{sensitive=false,
  basicstyle=\footnotesize\ttfamily,
  keywordstyle=\bfseries\color{keyword},
  keywords={@string,@article,@book,@misc,@proceedings,@manual,@modify,@alias,@include},
  backgroundcolor=\color{bibtex-bg},
  frame=single,
  framerule=0pt}
\lstdefinelanguage{BibTool}{sensitive=false,
  basicstyle=\scriptsize\ttfamily,
  keywordstyle=\bfseries\color{keyword},
  backgroundcolor=\color{rsc-bg},
  frame=single,
  framerule=0pt}

\makeatletter%>>>>>>>>>>>>>>>>>>>>>>>>>>>>>>>>>>>>>>>>>>>>>>>>>>>>>>>>>>>>>>>>>

\newcommand\opt[1]{\texttt{-{}#1}\index{#1@{\tt-{}#1}}}

\definecolor{sh-bg}{rgb}{.9,.9,.9}
\definecolor{sh-border}{rgb}{.4,.4,.4}

\newcommand\sh{\smallskip\par\hspace*{2em}\@ifnextchar[{\sh@}{\sh@@}}
\def\sh@[#1]#2{%
  \fcolorbox{sh-border}{sh-bg}{\hspace*{1em}\begin{minipage}{.85\textwidth}
      \texttt{bibtool -{}#1 }\textit{#2}\index{#1@{\tt-{}#1}}
    \end{minipage}}\smallskip\par\noindent\ignorespaces}
\newcommand\sh@@[1]{%
  \fcolorbox{sh-border}{sh-bg}{\hspace*{1em}\begin{minipage}{.85\textwidth}
      \texttt{bibtool }\textit{#1}
    \end{minipage}}\smallskip\par\noindent\ignorespaces}

\makeatother%<<<<<<<<<<<<<<<<<<<<<<<<<<<<<<<<<<<<<<<<<<<<<<<<<<<<<<<<<<<<<<<<<<

\newcommand\rsc[1]{\textsf{#1}\index{#1@\textsf{#1}}}
\newcommand\rscEqBraces[2]{\rsc{#1} = \(\{\)#2\(\}\)}
\newcommand\rscBraces[2]{\rsc{#1} \(\{\)#2\(\}\)}
\newcommand\rscIt[1]{\textsf{\textit{#1}}}
\newcommand\env[1]{\texttt{#1}\index{#1@{\tt #1}}}

\definecolor{rsc-bg}{rgb}{.9,1,.9}
\definecolor{rsc-border}{rgb}{.4,.6,.4}
\newbox\rscbox
\newenvironment{Resources}%
  {\setbox\rscbox=\hbox\bgroup\hspace*{.06\textwidth}\begin{minipage}{.88\textwidth}\sf\obeylines\obeyspaces
  }{\end{minipage}\egroup\vspace{1.5ex}%
    \fcolorbox{rsc-border}{rsc-bg}{\usebox\rscbox}%
    \medskip\par
  }

\newcommand\code[1]{\texttt{#1}}
\newcommand\file[1]{\textsf{#1}}

\newcommand\off{\textsf{off}}
\newcommand\on{\textsf{on}}

\newcommand\bs{\texttt{\symbol{"5C}}\ignorespaces}
\newcommand\BS{\(\backslash\)}
\newcommand\Hat{\^{}}

\newfont\cminch{cminch}
\ifx\chaptername\relax\else
\renewcommand\chaptername{\cminch}
\renewcommand\appendixname{\cminch}
\fi

\newcommand\rfill[1]{\leaders\hrule height #1\hfill}

\newenvironment{Summary}{\subsection*{Summary}
  \begin{center}\small
    \begin{tabular}{p{.08\textwidth}p{.35\textwidth}p{.48\textwidth}}
      \toprule
      \textit{\scriptsize Option}&
      \textit{\scriptsize Resource command}&
      \textit{\scriptsize Description}\\
      \midrule
}{\bottomrule\end{tabular}\end{center}}

\newcommand\Desc[3]{\textit{#1}&\textit{#2}&#3\\\hline}

\newenvironment{Example}{\smallskip\par\textit{Example}\par}{\smallskip\par}

\newenvironment{Disclaimer}{\begin{center}\sf\tiny --- }{ ---\end{center}}

\newcommand\LINK[2]{\texttt{#2}}
\newcommand\Link[2]{\href{#1}{\texttt{#2}}}
\newcommand\email[1]{\href{mailto:#1}{\texttt{#1}}}

\newcommand\FTP[2]{\texttt{#1}}

\newcommand\ASCII{\textsc{ascii}}

\begin{document} %%%%%%%%%%%%%%%%%%%%%%%%%%%%%%%%%%%%%%%%%%%%%%%%%%%%%%%%%%%%%%

\chapter{The \BibTool3 Configuration Language}

With version 3 of \BibTool\ the configuration languages has been extended. It
contains a rich set of possibilities which open new manipulations of \BibTeX\
files.

The extension has been inspired by functional programming as in Lisp and
object oriented programming as in C++.

\section{Basic Constructions}

% TODO

\subsection{Comments}

Comments are started by a percent sign and extend to the end of the line.

\begin{lstlisting}[language=BibTool]
% this is a comment
\end{lstlisting}

\subsection{Words}\ref{sec:words}

Words are sequences of characters which are used as variable names, function
names, method names, and class names. Words start with a lower case or upper
case letter, \verb|$|, \verb|@|, \verb|.|, or \verb|_|. They are optionally
followed by a letter, a digit, \verb|$|, \verb|@|, \verb|.|, or \verb|_|.

The follwoing words are reserved:

\verb|and|
\verb|defun|
\verb|each|
\verb|else|
\verb|false|
\verb|function|
\verb|if|
\verb|ilike|
\verb|like|
\verb|mod|
\verb|nil|
\verb|not|
\verb|or|
\verb|quote|
\verb|return|
\verb|true|
\verb|while|
\verb|with|

\subsection{Variables}

% TODO

\subsection{Statements}

% TODO

\subsection{Blocks}

% TODO

\subsection{Assignments}

\begin{lstlisting}[language=BibTool]
x = 123
\end{lstlisting}

\subsection{Variable Scopes}

% TODO

\begin{lstlisting}[language=BibTool]
with (x:123) {}
\end{lstlisting}

\subsection{Quote}


\section{Strings}

Strings are enclosed in double quotes. They may contain arbitrary characters.

\begin{lstlisting}[language=BibTool]
""
\end{lstlisting}

\begin{lstlisting}[language=BibTool]
"a b c"
\end{lstlisting}

The backslash (\verb|\|)acts as escape character.
\begin{description}
\item [\BS\BS] \ \\The doubled backslash is stored as single backslash
  internally.
\item [\BS "] \ \\The delimiting character \verb|"| can be included in a
  string by preceeding it with a backslash.
\item [\BS t] \ \\The tab character can be included in a string as the
  sequence \verb|\t|
\item [\BS n] \ \\The newline character can be included in a string as the
  sequence \verb|\n|
\item [\BS r] \ \\The carrige return character can be included in a string as
  the sequence \verb|\r|
\item [\BS b] \ \\The backspace character can be included in a string as the
  sequence \verb|\b|
\item [\BS f] \ \\The formfeed character can be included in a string as the
  sequence \verb|\n|
\end{description}

\subsection{String Comparisons}

%TODO


\subsection{String method \texttt{as.boolean()}}

The string method \texttt{as.boolean()} converts the string into a boolean
value. It is true if the string is not empty and false otherwise. This method
takes no arguments.

\begin{lstlisting}[language=BibTool,mathescape=true]
"123":as.boolean()
$\hookrightarrow$ true
\end{lstlisting}

\subsection{String method \texttt{as.string()}}

The string method \texttt{as.string()} converts the string into a string
value. This is the string itself. This method is provided for consistency.
This method takes no arguments.

\begin{lstlisting}[language=BibTool,mathescape=true]
"123":as.string()
$\hookrightarrow$ "123"
\end{lstlisting}

\subsection{String method \texttt{as.number()}}

The string method \texttt{as.number()} converts the string into a number
value. The string is scanned and the first sequence consisting of digits
optionally preceeded by a minus sign are returned as number. If no number is
found then 0 is returned. This method takes no arguments.

\begin{lstlisting}[language=BibTool,mathescape=true]
"abc 123 def ":as.number()
$\hookrightarrow$ 123
\end{lstlisting}

\subsection{String method \texttt{class()}}

The string method \texttt{class()} returns the class object for strings. This
method takes no arguments. More on classes can be found in
section~\ref{sec:classes}.

\begin{lstlisting}[language=BibTool,mathescape=true]
"123, 456":class()
$\hookrightarrow$ <Class String>
\end{lstlisting}

\subsection{String method \texttt{concat()}}

The string method \texttt{concat()} concatenates its arguments to the end of
the current string and returns the new string as result.
This method takes arbitrarily many arguments.

\begin{lstlisting}[language=BibTool,mathescape=true]
"123":concat("a b c")
$\hookrightarrow$ "123a b c"
\end{lstlisting}

\subsection{String method \texttt{length()}}

The string method \texttt{length()} returns the number of characters contained
in the string. This method takes no arguments.

\begin{lstlisting}[language=BibTool,mathescape=true]
"123 456":length()
$\hookrightarrow$ 7
\end{lstlisting}

%TODO

\subsection{Iterating over Strings}

%TODO

\begin{lstlisting}[language=BibTool,mathescape=true]
each (c:"123 456") {...}
\end{lstlisting}

\section{Numbers}

Numbers can be decimal numbers made up of digits and an optional minus prefix.

\begin{lstlisting}[language=BibTool]
123
\end{lstlisting}

\begin{lstlisting}[language=BibTool]
-987
\end{lstlisting}

Numbers can be hexadecimal numbers, Those start with \verb|0x| and continue
with digits and the letters A to F ot a to f. An optional minus may prefix the
number.

\begin{lstlisting}[language=BibTool]
0x1234
\end{lstlisting}

Numbers can be octal numbers, Those start with \verb|0| and continue with
digits 0 to 7. An optional minus may prefix the number.

\begin{lstlisting}[language=BibTool]
0123
\end{lstlisting}

\subsection{Numeric Expressions}

Numeric expressions can be made up of the infix operators for plus (\verb|+|),
minus (\verb|-|), times (\verb|*|), divide (\verb|/|), and mod (\verb|mod|).
In addition the unary minus as prefix operator can be used, as well as
parentheses to determine the priority of the evaluation.

The evaluation is performed left to right, unless forced differntly by the
prioities of the operators. Any operator involved is converted to a number for
the evaluation. This is done with the built-in version of the appropriate
method \verb|as.number()|.

%TODO

\begin{lstlisting}[language=BibTool,mathescape=true]
1 + 2
$\hookrightarrow$ 3
\end{lstlisting}

\begin{lstlisting}[language=BibTool,mathescape=true]
1 - 2
$\hookrightarrow$ -1
\end{lstlisting}

\begin{lstlisting}[language=BibTool,mathescape=true]
-(1 + 2)
$\hookrightarrow$ -3
\end{lstlisting}

\begin{lstlisting}[language=BibTool,mathescape=true]
2 * 3
$\hookrightarrow$ 6
\end{lstlisting}

\begin{lstlisting}[language=BibTool,mathescape=true]
4 / 2
$\hookrightarrow$ 2
\end{lstlisting}

\begin{lstlisting}[language=BibTool,mathescape=true]
12 mod 5
$\hookrightarrow$ 2
\end{lstlisting}

\subsection{Numeric Comparisons}

%TODO

\subsection{Number method \texttt{class()}}

The number method \texttt{class()} returns the class object for numbers. This
method takes no arguments. More on classes can be found in
section~\ref{sec:classes}.

\begin{lstlisting}[language=BibTool,mathescape=true]
456:class()
$\hookrightarrow$ <Class Number>
\end{lstlisting}

%TODO

\section{Booleans}

Booleans are represented with the built in constants \verb|true| and
\verb|false|.

\begin{lstlisting}[language=BibTool]
true
\end{lstlisting}

\begin{lstlisting}[language=BibTool]
false
\end{lstlisting}

%TODO

\subsection{Boolean Operators}

\subsubsection{The Conjunction}

The conjunction of boolean terms is denoted with the infix operator \verb|and|
or \verb|&&|. The two notations are identical and interchangable.

The evaluation is performed left to right. Shortcut-evaluation ist used. This
means the evaluation stops as soon as the result can be inferred.

\begin{lstlisting}[language=BibTool]
a and b
\end{lstlisting}

\begin{lstlisting}[language=BibTool]
a && b
\end{lstlisting}


\subsubsection{The Disjunction}

The disjunction of boolean terms is denoted with the infix operator \verb|or|
or \verb'||'. The two notations are identical and interchangable.

The evaluation is performed left to right. Shortcut-evaluation ist used. This
means the evaluation stops as soon as the result can be inferred.

\begin{lstlisting}[language=BibTool]
a or b
\end{lstlisting}

\begin{lstlisting}[language=BibTool]
a || b
\end{lstlisting}

\subsubsection{The Negation}

The negation of a boolean term is denoted with the prefix operator \verb|not|
or \verb|!|. The two notations are identical and interchangable.

The evaluation is performed left to right. Shortcut-evaluation ist used. This
means the evaluation stops as soon as the result can be inferred.

\begin{lstlisting}[language=BibTool]
not a
\end{lstlisting}

\begin{lstlisting}[language=BibTool]
! a
\end{lstlisting}

\subsubsection{Boolean Comparisons}



%TODO

\subsection{Boolean method \texttt{as.boolean()}}
n method \texttt{as.boolean()} converts the boolean into a boolean
value. This is the boolean itself. This method is provided for consistency.
This method takes no arguments.

\begin{lstlisting}[language=BibTool,mathescape=true]
true:as.boolean()
$\hookrightarrow$ true
\end{lstlisting}

\subsection{Boolean method \texttt{as.number()}}

The boolean method \texttt{as.number()} converts the boolean into a numeric
value. The value 1 for \verb|true| and 0 for \verb|false|. This method takes
no arguments.

\begin{lstlisting}[language=BibTool,mathescape=true]
true:as.number()
$\hookrightarrow$ 1
\end{lstlisting}

\begin{lstlisting}[language=BibTool,mathescape=true]
false:as.number()
$\hookrightarrow$ 0
\end{lstlisting}

\subsection{Boolean method \texttt{as.string()}}

The boolean method \texttt{as.number()} converts the boolean into a string
value. The value is the printable representation of the boolean. This method
takes no arguments.

\begin{lstlisting}[language=BibTool,mathescape=true]
true:as.string()
$\hookrightarrow$ "true"
\end{lstlisting}

\begin{lstlisting}[language=BibTool,mathescape=true]
false:as.string()
$\hookrightarrow$ "false"
\end{lstlisting}

\subsection{Boolean method \texttt{class()}}

The boolean method \texttt{class()} returns the class object for booleans.
This method takes no arguments. More on classes can be found in
section~\ref{sec:classes}.

\begin{lstlisting}[language=BibTool,mathescape=true]
true:class()
$\hookrightarrow$ <Class Boolean>
\end{lstlisting}

%TODO

\section{Lists}

Lists are denoted by values enclosed in brackets and separated by commas.
Thus the empty list is denoted by empty brackets:

\begin{lstlisting}[language=BibTool,mathescape=true]
[]
\end{lstlisting}

A single element list can be seen in the following example:

\begin{lstlisting}[language=BibTool,mathescape=true]
[123]
\end{lstlisting}

\begin{lstlisting}[language=BibTool,mathescape=true]
[123, 456]
\end{lstlisting}


\subsection{List Comparisons}

%TODO


\subsection{List method \texttt{as.boolean()}}

The list method \texttt{as.boolean()} converts the list into a boolean value.
It is true if the list is not empty and false otherwise. This method takes no
arguments.

\begin{lstlisting}[language=BibTool,mathescape=true]
[123, 456]:as boolean()
$\hookrightarrow$ true
\end{lstlisting}

\subsection{List method \texttt{as.number()}}

The list method \texttt{as.number()} converts the list into a numeric value.
The value is the length of the list. This method takes no arguments.

\begin{lstlisting}[language=BibTool,mathescape=true]
[123, 456]:as.number()
$\hookrightarrow$ 2
\end{lstlisting}

\subsection{List method \texttt{as.string()}}

The list method \texttt{as.number()} converts the list into a string value.
The value is the printable representation of the list. This method takes no
arguments.

\begin{lstlisting}[language=BibTool,mathescape=true]
[123,456]:as.string()
$\hookrightarrow$ "[123, 456]"
\end{lstlisting}

\subsection{List method \texttt{car()}}

The list method \texttt{car()} returns the first element of the list. If the
list is empty then the empty list is returned. This method takes no arguments.

\begin{lstlisting}[language=BibTool,mathescape=true]
[123, 456]:car()
$\hookrightarrow$ 123
\end{lstlisting}

\subsection{List method \texttt{cdr()}}

The list method \texttt{cdr()} returns the list excluding the first element.
If the list is empty then the empty list is returned. This method takes no
arguments.

\begin{lstlisting}[language=BibTool,mathescape=true]
[123, 456]:cdr()
$\hookrightarrow$ [456]
\end{lstlisting}

\subsection{List method \texttt{class()}}

The list method \texttt{class()} returns the class object for lists. This
method takes no arguments. More on classes can be found in
section~\ref{sec:classes}.

\begin{lstlisting}[language=BibTool,mathescape=true]
[123, 456]:class()
$\hookrightarrow$ <Class List>
\end{lstlisting}

\subsection{List method \texttt{empty()}}

The list method \texttt{empty()} returns the boolean value true if the list is
empty and false otherwise. This method takes no arguments.

\begin{lstlisting}[language=BibTool,mathescape=true]
[123, 456]:empty()
$\hookrightarrow$ false
\end{lstlisting}

\subsection{List method \texttt{join()}}

The list method \texttt{join()}
% TODO

\begin{lstlisting}[language=BibTool,mathescape=true]
[123, 456]:join([9,8,7])
$\hookrightarrow$ [123, 456, 9, 8, 7]
\end{lstlisting}

\subsection{List method \texttt{length()}}

The list method \texttt{length()} returns the number of elements contained in
the list. This method takes no arguments.

\begin{lstlisting}[language=BibTool,mathescape=true]
[123, 456]:length()
$\hookrightarrow$ 2
\end{lstlisting}

\subsection{Iterating over Lists}

%TODO

\section{Functions and Methods}

%TODO


\subsection{Function method \texttt{apply()}}

The function method \texttt{apply()} runs the code associated to the funtion
after the parameters are bound to the actual values or the fallbacks defined
with the function.

\begin{lstlisting}[language=BibTool,mathescape=true]
function (x:123){ x }:apply()
$\hookrightarrow$ 123
\end{lstlisting}

\begin{lstlisting}[language=BibTool,mathescape=true]
function (x:123){ x }:apply(987)
$\hookrightarrow$ 987
\end{lstlisting}


\subsection{Function method \texttt{class()}}

The function method \texttt{class()} returns the class object for functions.
This method takes no arguments.

\begin{lstlisting}[language=BibTool,mathescape=true]
function(){}:class()
$\hookrightarrow$ <Class Function>
\end{lstlisting}

\subsection{Function method \texttt{code()}}

The function method \texttt{code()} returns the code as list. This method
takes no arguments.

\begin{lstlisting}[language=BibTool,mathescape=true]
function (x:123){ x }:code()
$\hookrightarrow$ [x]
\end{lstlisting}


\section{Classes}\label{sec:classes}

%TODO

\subsection{Class method \texttt{class()}}

The class method \texttt{class()} returns the class object for classes. This
method takes no arguments.

\begin{lstlisting}[language=BibTool,mathescape=true]
"123, 456":class():class()
$\hookrightarrow$ <Class Class>
\end{lstlisting}


\section{Databases}

\begin{lstlisting}[language=BibTool,mathescape=true]
with (db:read("abc.bib")) {
  % do something with the db
}
\end{lstlisting}

\subsection{The function \texttt{read()}}\label{db:read}

The function \texttt{read()} creates a new database and optionally reads in
some files. The arguments can be of several kinds.

If no arguments are given then no files are read.

\begin{lstlisting}[language=BibTool,mathescape=true]
db = read()
\end{lstlisting}

If an argument is a string then theis string is taken as file name to read
from. The resource \rsc{bibtex.search.path} is used to determine the
directories to consider. The file extension \verb|.bib| is appended if the raw
file does not exist.

If the file can not be opened then an error is raised.

\begin{lstlisting}[language=BibTool,mathescape=true]
db = read("abc.bib")
\end{lstlisting}

Several arguments can be given.

\begin{lstlisting}[language=BibTool,mathescape=true]
db = read("abc.bib", "def.bib")
\end{lstlisting}

If an argument is a list of strings then those files are read in turn.

\begin{lstlisting}[language=BibTool,mathescape=true]
files = ["abc.bib", "def.bib"]
db = read(files)
\end{lstlisting}

\subsection{Database method \texttt{class()}}

The database method \texttt{class()} returns the class object for databases.
This method takes no arguments. More on classes can be found in
section~\ref{sec:classes}.

\begin{lstlisting}[language=BibTool,mathescape=true]
db:class()
$\hookrightarrow$ <Class Database>
\end{lstlisting}

\subsection{Database method \texttt{read()}}

The database method \texttt{read()} reads additional \BibTeX\ files into the
database. The arguments are the same as for the function \rsc{read} (see
section~\ref{db:read}).

The return value of this method is the database. Thus method invocations can
be cascaded.

\begin{lstlisting}[language=BibTool,mathescape=true]
db:read("abc.bib")
\end{lstlisting}

\subsection{Database method \texttt{sort()}}

The database method \texttt{sort()} sorts the normal records of the
database. The default is to sort case-insensitive according to the sort key in
ascending order.

\begin{lstlisting}[language=BibTool,mathescape=true]
db:sort()
\end{lstlisting}

The sort order can be influenced optional arguments:

\begin{itemize}
\item If the argument is the string \verb|"asc"|, the string
  \verb|"ascending"|, the word \verb|'asc|, or the word \verb|'ascending| then
  the records are sorted in ascending order.
\begin{lstlisting}[language=BibTool,mathescape=true]
db:sort('asc)
\end{lstlisting}

\item If the argument is the string \verb|"desc"|, the string
  \verb|"descending"|, the word \verb|'desc|, or the word \verb|'descending|
  then the records are sorted in descending order.
\begin{lstlisting}[language=BibTool,mathescape=true]
db:sort('desc)
\end{lstlisting}

\item If the argument is the string \verb|"case"|, or the word \verb|'case|
  then the records are sorted in case-sensitive. The default is to be not case
  sensitive. This flag can not be combined with a sorting function.
\begin{lstlisting}[language=BibTool,mathescape=true]
db:sort('case)
\end{lstlisting}

\item If the argument is a list then the arguments described above can be
  contained and have the same effect as single arguments.
\begin{lstlisting}[language=BibTool,mathescape=true]
db:sort( ['asc, 'case] )
\end{lstlisting}

\item If the argument is a function then this function determined to order of
  the records. The function takes two arguments which are records. It returns
  a boolean or number which is 0 if the second record should go before the
  first record. The additional argument \verb|'desc| inverts this order. 
\begin{lstlisting}[language=BibTool,mathescape=true]
db:sort(function(r1,r2){return true})
\end{lstlisting}
\end{itemize}

The return value of this method is the database. Thus method invocations can
be cascaded.


\subsection{Database method \texttt{write()}}

The database method \texttt{write()} writes a database to a given file.
This method takes two optional arguments.

\begin{lstlisting}[language=BibTool,mathescape=true]
db:write()
\end{lstlisting}

The first argument denotes the name of the output file. If the output file is
not given or is empty then the output is directed to the standar output
stream. If the output file can not be opened then an error is raised.

\begin{lstlisting}[language=BibTool,mathescape=true]
db:write("abc.bib")
\end{lstlisting}

The second optional argument is a string which specifies sections of the
database to be written and their sequence. The following characters can be
contained in the specification:

\begin{description}
\item [p] The preamble.
\item [\$] All strings (macros) contained in the database.
\item [S] The strings (macros) which are used in the database.
\item [s] The strings (macros) contained in the database where the resource
  print.all.strings determines whether all strings should be printed or the
  used strings only.
\item [n] The normal records.
\item [c] The comments.
\item [i] The includes.
\item [a] The aliases.
\item [m] The modifies.
\end{description}
Upper-case letters which are not mentioned are silently folded to their
lower-case counterparts.

The default if not given then the default is taken from the variable
\rsc{print.entry.types}.

The return value of this method is the database. Thus method invocations can
be cascaded.

\subsection{Iterating over Databases}

%TODO

\begin{lstlisting}[language=BibTool,mathescape=true]
each (e:db) {...}
\end{lstlisting}

\section{Entries}

%TODO

\subsection{Entry method \texttt{class()}}

The entry method \texttt{class()} returns the class object for entries. This
method takes no arguments. More on classes can be found in
section~\ref{sec:classes}.

\begin{lstlisting}[language=BibTool,mathescape=true]
entry:class()
$\hookrightarrow$ <Class Entry>
\end{lstlisting}

\subsection{Iterating over Entries}

%TODO

\begin{lstlisting}[language=BibTool,mathescape=true]
each (x:entry) {...}
\end{lstlisting}


\end{document} %%%%%%%%%%%%%%%%%%%%%%%%%%%%%%%%%%%%%%%%%%%%%%%%%%%%%%%%%%%%%%%%
%
% Local Variables:
% mode: latex
% TeX-master: nil
% fill-column: 78
% End:
