%%*** install.tex **************************************************************
%% 
%% This file is part of BibTool.
%% It is distributed under the Creative Commons Attribution-Share
%% Alike 3.0 License.
%% 
%% (c) 2010-2014 Gerd Neugebauer
%% 
%% Net: gene@gerd-neugebauer.de
%% 
%%*****************************************************************************
\documentclass[11pt,a4paper]{scrartcl}

\usepackage[latin1]{inputenc}
\usepackage{hyperref}

\newcommand\code[1]{\texttt{#1}}
\newcommand\file[1]{\textsf{#1}}

\newcommand\BS{\(\backslash\)}

\newcommand\BibTool{{\sc Bib\hskip-.1em\-%
        \mbox{T\hskip-.15emo\hskip-.05emo\hskip-.05eml}}}
\newcommand\BibTeX{{\rm B\kern-.05em{\sc i\kern-.025em b}\kern-.08em
        T\kern-.1667em\lower.7ex\hbox{E}\kern-.125emX}}

\date{}

\begin{document}%%%%%%%%%%%%%%%%%%%%%%%%%%%%%%%%%%%%%%%%%%%%%%%%%%%%%%%%%%%%%%%

\title{\BibTool{} Installation}
\author{Gerd Neugebauer}
\maketitle

\BibTeX{} provides an easy to use means to integrate citations and
bibliographies into \LaTeX{} documents.  But the user is left alone
with the management of the \BibTeX{} files. The program BibTool is
intended to fill this gap.  \BibTool{} allows the manipulation of
\BibTeX{} files which goes beyond the possibilities --- and intentions
--- of \BibTeX{}.  The possibilities of \BibTool{} include

\begin{itemize}
  
\item Pretty-printing \BibTeX{} data bases adjustable by lots of
  parameters.
  
\item Syntactic checks with error recovery superior to \BibTeX{} and
  helpful error messages.

\item Semantic checks can be specified by the user.
  
\item Sorting and merging of \BibTeX{} data bases according to a free
  definable sort key.
  
\item Generation of uniform reference keys according to predefined
  rules or according to an own specification.
  
\item Selecting references used in one publication which are found by
  analyzing an .aux file.
  
\item Selecting references by a set of criteria (regular expressions).
  
\item Controlled rewriting of fields utilizing regular expressions to
  specify the rewriting rules.
  
\item Macro (String) expansion to eliminate the need of extra string
  definitions.
  
\item Collecting statistics about one or more \BibTeX{} data bases.
\end{itemize}

\BibTool{} contains a documentation written in \LaTeX{} of more than
60 pages (and still growing).

\BibTool{} is written in C and has been compiled on various operating
systems like flavors of UN*X and MSDOG machines.  It is distributed in
source code.  No compiled versions are available from the author
(Useless to ask!).

\BibTool{} can be obtained from the CTAN archives:

\begin{itemize}
\item
  \url{http://www.ctan.org/tex-archive/biblio/bibtex/utils/bibtool}
\end{itemize}

Get  the  file BibTool-x.xx.tar.gz  where  x.xx is the version number.
Unpack it with the command (on UN*X)

\begin{verbatim}
	gunzip < BibTool-x.xx.tar.gz | tar -xvf -
\end{verbatim}

It will create a directory named BibTool-x.xx which contains the
installation instructions in the file \file{install.tex} and
\file{INSTALL}.


%This file contains instructions and hints for the compilation of \BibTool{}.
\newpage
\tableofcontents
\newpage

\section{Introduction}

\BibTool{} is known to work on the following operating systems/C
compilers: (At least an earlier version has been compiled successfully
on them.)

\begin{description}
\item[Debian]		gcc
\item[CentOS]		gcc
\item[Ubuntu]		gcc
\item[SuSE]		gcc
\item[Windows XP/Cygwin]	gcc
\item[Windows 7/Cygwin 1.7]	gcc
\item[Sun4/SunOS4.1.*]		cc, gcc
\item[SparcStation/Solaris2.4 gcc
\item[Sun3/SunOS4.1.1]		cc
\item[HP 9000/4??/HP UX ??]	cc
\item[Atari ST/TOS 1.2]	Laser C, Pure C
\end{description}
        
I have been informed that \BibTool{} has been compiled on the
following machines/operating systems (and maybe some more that I have
forgotten to add to this list;-):

\begin{description}
\item[OS X Mavericks]
\item[AIX 3.2.5]		gcc
\item[AIX 4.1.4]
\item[Amiga]			SAS/C
\item[DEC Alpha/OSF/1]		
\item[HP 9000/778/HP-UX 10.20] gcc 2.7.2.3
\item[MIPS/Ultrix4.4]		gcc 2.4.5
\item[MIPS/RISCos 4.52]		gcc
\item[MSDOS,OS/2,Win32,WinNT]  emx, dj, Watcom C, MSC
\item[MS-Windows 4.0, Sp 6]	Cygwin
\item[IBM PC] 486DX2-66, OS/2 Merlin 4.0 FixPack \#5. EMX v.0.9c+GCC+dmake 4.0
\item[NeXT]			
\item[SGI Indigo2/IRIX 6.2]	cc
\item[Windows 7/Cygwin 1.7]	gcc
\item[Windows XP 32 bits]	MinGW with GCC 4.6.1
\end{description}

If you compile \BibTool{} on other systems drop me a mail. I'm
interested to support a broad variety of systems/compilers.

\begin{quote}
  Gerd Neugebauer\\
  Im Lerchelsb�hl 5\\
  64521 Gro�-Gerau (Germany)
  
  Net: \url{gene@gerd-neugebauer.de}
\end{quote}

You can also send me a mail if you encounter problems in compiling
\BibTool{}, or crashes, or shows unexpected (contradicting the
documentation) behavior of the final program.

Please enclose a precise description what went wrong.  Include the
version numbers of \BibTool{}, your computer, your operating system,
and your C compiler.

If you encounter unexpected behavior of \BibTool{} enclose your
resource file(s) and a \emph{small} \BibTeX{} file demonstrating the
problem. Describe how you have invoked \BibTool{} (arguments) and
justify why you think that there is a problem.

I know that this sounds like work.  But otherwise I will not be able
to give you proper advice or find a bug in the sources.


\section{Prerequisites}

To install \BibTool{} you need:

\begin{itemize}
\item A C compiler. ANSI-C is not required but highly recommended.
  Several library functions are expected. You will see if your linker
  complains.
\item A running version of make is recommended.
\end{itemize}


\section{Generic Installation Procedure}\label{generic}

See if there is a special section for your computer/C compiler and
follow the instructions given there.  Most probably you will be
directed back into this section after some initial actions.

See section~\ref{kpathsea} and see if it applies to your setting.

To install \BibTool{} you can try to apply the following instructions:

\begin{enumerate}
\item Copy one of the prepared makefiles to \file{makefile} and adjust
  it according to your C compiler.  The following makefiles should be
  present in the distribution:

  \begin{description}
  \item[makefile.unx] This is the generic makefile. Especially it is
    for all UN*X systems and may provide a basis for compilation on
    systems not explicitly mentioned here.
  \item[makefile.dos]	This is the makefile for MSDOS based machines.
  \item[makefile.ata]	This is the makefile for Atari ST/TT
    computers.
  \item[makefile.ami]	This is the makefile for Amiga computers (with
    SAS/C).        
  \end{description}
  
  Edit the \file{makefile} and adjust the settings in the
  configuration section according to your needs.
  
  Maybe you have to adapt the name and options for your C compiler.
  (Most of the time I prefer the GNU C compiler)
  
  Maybe you want to change the location for the installation target,
  even though reasonable defaults are provided.

\item Look into the file \file{include/bibtool/config.h} to see if it
  fits your operating system and C compiler.  Adjust things as
  required.  Most probably you do not have to change anything in this
  file.  This file is mainly for those with an old C compiler
  (non-ANSI).
  
\item If you want to configure some of the internals of \BibTool{} you
  can have a look into \file{include/bibtool/bibtool.h.}  It contains
  the internal configuration options which used to be in
  \file{config.h } prior to release 2.39.  Most probably there is no
  need to change the defaults given there.  One exception is the
  support for another language by specifying additional words to be
  ignored.  (Maybe I could include them into the distribution; thus
  drop me a mail if you do so)
  
\item If you have a working makedepend command run
\begin{verbatim}
	make depend 
\end{verbatim}
  
  Otherwise just skip this step. It is only helpful if you compile
  \BibTool{} more than once.
  
  This might cause problems if the makedepend command does not fit the
  C compiler used (e.g.  proprietary makedpend together with gcc).
  This results in unknown files to show up.  In this case also skip
  this step and revert to the original makefile.
  
\item Afterwards run

\begin{verbatim}
	make
\end{verbatim}
  
  Most probably this should produce the executable \file{bibtool} (or
  \file{bibtool.exe}, or \file{bibtool.ttp}, or \ldots) in the current
  directory.
  
\item If you have tried the command bibtool in the current directory
  you can install it with

\begin{verbatim}
	make install
\end{verbatim}
    
  to install the executable and the libraries and

\begin{verbatim}
	make install.man
\end{verbatim}
    
  to install the UN*X man pages. This is only useful if you can make
  use of them, i.e. on a UN*X-like system.
    
\item To get rid of all intermediate files run

\begin{verbatim}
	make clean
\end{verbatim}
  
\item To prepare the documentation contained in the sub-directory
  \file{doc} go into the \file{doc} subdirectory and run

\begin{verbatim}
	latex   bibtool
	latex   bibtool
	bibtex  bibtool
	makeidx bibtool
	latex   bibtool
	latex   bibtool
\end{verbatim}

\begin{verbatim}
	latex   ref_card
	latex   ref_card
	bibtex  ref_card
	makeidx ref_card
	latex   ref_card
	latex   ref_card
\end{verbatim}
  
  The makeidx program may be named differently on your system or
  missing at all. In this case you can omit this step and do without
  the index.
  
  On UN*X you can try

\begin{verbatim}
	make doc
\end{verbatim}
  
  in the main directory or

\begin{verbatim}
        make
\end{verbatim}
  
  in the \file{doc} directory which tries to perform the steps given
  above.
  
  The documentation is written in a way that either \LaTeX{} or
  \LaTeX2.09 can be used to compile it. I had a bug report about a
  real ancient \LaTeX2.09 and I am not sure I have completely fixed
  it.  Thus the best is to use an up-to-date \LaTeX.
  
  There might be problems when files produced by \LaTeX2.09 should be
  read by \LaTeXe{} and vice versa.  In this case you can try to
  remove the intermediate files

\begin{verbatim}
	bibtool.toc
	bibtool.ind
\end{verbatim}
  
  and follow the instruction for making the documentation from the
  beginning.

\end{enumerate}


\section{Installation on UNIX}

Some special preparations have been made for UNIX systems.  autoconf scripts
are  provided which  try  to find  out the   characteristics  of  the system
automatically.

General instructions:

\begin{enumerate}
\item 
    Run the configure script with the command:

\begin{verbatim}
	./configure
\end{verbatim}

    The following options are useful for configure:

    \begin{description}
    \item [--with-kpathsea]	 try to find the kpathsea library
      either in the current directory or in the previous directory.
      Thus you can place the \BibTool{} directory in the same
      directory where the directory kpathsea resides or you can put
      the kpathsea directory in in the \BibTool{} directory.
      (default) (See also section~\ref{kpathsea})
    \item [--without-kpathsea]	 disable the search for the kpathsea
      library.
    \item [--prefix=PREFIX] install architecture-independent files in
      PREFIX [/usr/local]
    \item [--exec-prefix=EPREFIX] install architecture-dependent files
      in EPREFIX [same as prefix]
    \item [--bindir=DIR] the bibtool executable is stored in DIR
      [EPREFIX/bin]
    \item [--libdir=DIR] the \BibTool{} directory containing certain
      resource files are stored in DIR [EPREFIX/lib]
    \item [--mandir=DIR] the manual pages bibtool.1 is stored in DIR
      [PREFIX/man]
    \end{description}

\item 
    Continue with the item C in section \ref{generic}.
\end{enumerate}


\section{Installation on MSDOS-like Computers}

This  section some hints on  the compilation of  \BibTool{} on MSDOS computers.
These adaptions   are  mainly  due   to  the   efforts  of   Josef  Spangler
(JS@rphnw3.ngate.uni-regensburg.de). All credits  go to him.  Any  remaining
problems should be blamed on my ignorance.

General instructions:

\begin{enumerate}
\item Copy \file{makefile.dos} to \file{makefile} and adjust it
  according to your C compiler.
  
\item Edit \file{include/bibtool/config.h} and
  \file{include/bibtool/bibtool.h} to adjust it to your needs.
  Normally you should be interested only in the support of em\TeX{} at
  the end of \file{include/bibtool/bibtool.h}.
  
\item Copy MSDOS\BS link.* to the source directory.
  
\item Depending on the C compiler do the following:

    \begin{itemize}
    \item dj (GNU C port)

\begin{verbatim}
	make dj
\end{verbatim}

    \item emx (GNU C port)

\begin{verbatim}
	make emx
\end{verbatim}

      Maybe you have to set the environment variable EMX to point to the 
      installed EMX directory.

    \item Watcom C 386 (32 Bit Compiler for Dos, OS/2 2.x, Win32 and WinNT)

\begin{verbatim}
	make wat
\end{verbatim}

    \item Microsoft C 6.00A (16 Bit Dos and OS/2 1.x)

\begin{verbatim}
	make msc
\end{verbatim}

    \item Borland C++ 3.1 (16 Bit Dos)

	Go and get another compiler. 
	Or, even better do the port and send me the diffs :-)
    \end{itemize}

\item 
    If you have tried the command bibtool in the current directory
    you can install it with

\begin{verbatim}
	make install
\end{verbatim}

\item 
    Make the documentation according to step H in section \ref{generic}.
\end{enumerate}


\section{Installation on Amiga}

This section contains some hints on the compilation of \BibTool{} on
Amiga computers. These adaptions are mainly due to the efforts of
Andreas Scherer (SCHERER@genesis.informatik.rwth-aachen.de).  All
credits go to him.  Any remaining problems should be blamed on my
ignorance.

This section describes the installation procedure if you are using the
SAS/C compiler on Amiga.

General instructions:

\begin{enumerate}
\item Copy \file{makefile.ami} to \file{SMakefile} and adjust it
  accordingly.
    
\item Edit the file \file{include/bibtool/config.h} and
  \file{include/bibtool/bibtool.h} and adjust it to your needs.  There
  should not be much for you to change.

\item 
\begin{verbatim}
    	make
\end{verbatim}

\item 
    If you have tried the command bibtool in the current directory
    you can install it with

\begin{verbatim}
	make install
\end{verbatim}

\item 
    Make the documentation according to step H in section \ref{generic}.
\end{enumerate}
 

\section{Installation without make}

\begin{enumerate}
\item Adjust the settings in the files \file{include/bibtool/config.h}
  and \file{include/bibtool/bibtool.h} to fit your C compiler and
  operating system.
  
\item Compile all .c files in the base directory.  The macros REGEX
  and maybe MSDOS should be defined.  The subdirectory
  \file{regex-0.12} should be included in the include search path.  A
  typical compile command looks like

\begin{verbatim}
	cc main.c -c -o main.o -DREGEX -DMSDOS -Iregex-0.12 -I.
\end{verbatim}
        
\item Compile \file{regex.c} in the regex-0.12 subdirectory and move
  the object file into the base directory (The one containing this
  file).
  
\item Link together all object files to get the executable bibtool.
  (maybe include the kpathsea library; see section \ref{kpathsea})
  
\item Run \LaTeX{} on \file{bibtool.tex} in the \file{doc}
  sub-directory to produce the documentation.  (see step H in section
  \ref{generic} for details)
\end{enumerate}


\section{Using the kpathsea library}\label{kpathsea}

The kpathsea library provides a very flexible way to specify the
search path for files.  This library is already used by the web2c
distribution of Karl Berry and the te\TeX{} distribution.  To achieve
compatibility with those \TeX{} systems you can try to make \BibTool{}
use the same library.


The kpathsea library is currently NOT CONTAINED in the distribution of
\BibTool{} since this is an experimental feature.  You can get
kpathsea from the same location where you got \BibTool{}.  Fetch the
file \file{kpse2-6.tar.gz} or \file{kpse2-6.zip}.

Unpack this file.  Enter the directory kpse2-6 and use the commands
`./configure' and `make' to create the library. This works at least on
Unix platforms. For porting to other platforms I would like to get
some feedback.

ATTENTION: The kpathsea library has to exists before \BibTool{} is
made!

If you are using configure then the option \texttt{--with-kpathsea}
enables the inclusion of the kpathsea routines if the library is
found.

To use the library you can just provide three definitions in the
makefile. See the section about kpathsea in \file{makefile} for
explanations. Alternatively you have to provide the appropriate
options to the C compiler yourself.

Note: the searching for \BibTeX{} files with the kpathsea library is
different from the \BibTool{} built-in searching. Some resources are
no longer taken into account when this library is used.

Note: kpathsea is only used to search for \BibTeX{} files.  All other
files are still searched with the traditional \BibTool{} searching
mechanism.  This might change soon.


\section{Problems and Porting}

Well, if the procedures described above don't work I have some hints.
These hints may also be useful if you plan to port \BibTool{} to other
operating systems or C compilers.

First of all a small list of assumptions that I use.

\begin{itemize}
\item On ANSI systems there should be no problem at all. There is some
  support for non-ANSI systems.  This support can be improved.  (I
  don't know if it's worthwhile to do so).
  
\item \BibTool{} has been developed on UN*X (SunOS) and UN*X-like
  systems (Atari).  Such file/directory naming conventions found their
  way into the code.  Single character delimiters between directories
  and files can be modified in a resource file.
  
\item ASCII encoding is assumed.  I role my own \file{type.h} instead
  of using \file{ctypes.h}.  This has mainly historical reasons. Maybe
  this code should be rewritten to be adapted at make time.
  
  If you port \BibTool{} to a non-ASCII machine the table of
  characters (allowed[]) in \file{type.h} has to be adapted -- at
  least (please send me the diffs).
  
\item Maybe some BSD-isms found their way into the code even if I am
  not aware of it.
\end{itemize}


\section{The C Interface}

The C interface to \BibTool{} is described in the document
\file{doc/c\_lib.dvi}.  To create this document run \LaTeX{} and
makeindex on the file \file{c\_lib.tex} in the \file{doca}
sub-directory:

\begin{verbatim}
	latex c_lib
	makeindex c_lib
	latex c_lib
	latex c_lib
\end{verbatim}

Read this document. I would like some kind of feedback for this attempt.

Attention: This documentation is experimental.  Some things are likely to be
	   changed soon.


\section{ The Tcl/Tk Interface}

The Tcl/Tk interface is called BibTcl.  It is contained in the BibTcl
subdirectory. See the file \file{README} in this directory for
details.



\end{document}%%%%%%%%%%%%%%%%%%%%%%%%%%%%%%%%%%%%%%%%%%%%%%%%%%%%%%%%%%%%%%%%%
%
% Local Variables: 
% mode: latex
% TeX-master: nil
% End: 
